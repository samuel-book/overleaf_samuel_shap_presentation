\begin{frame}
\frametitle{What `accuracy' hides - an example from the US legal system}
\small
We all know accuracy is not a perfect score, but let's look at this statement:

\vspace{2mm}
\textbf{\textit{`The COMPAS tool (used to predict re-offence rates in the US) is equally accurate in black and white men'.}
}
\vspace{3mm}

%\pause

Sounds good. And it is true (actually the accuracy was a little higher for black men), but at the same time....
\begin{itemize}
    \item Black defendants were predicted to be at a higher risk of re-offending than they actually were. 
    \item White defendants were predicted to be at a lower risk of re-offending than they actually were.
    \item Black defendants were twice as likely as white defendants to be misclassified as re-offenders.
    \item The error rate for black and white men was similar, but the error tended to be in opposite directions.
\end{itemize}

\vspace{3mm}
\tiny
\url{https://www.propublica.org/article/how-we-analyzed-the-compas-recidivism-algorithm}

\end{frame}