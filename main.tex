%%%%%%%%%%%%%%%%%%%%%%%%%%%%%%%%%%%%%%%%%%%%%%%%%%%%%%%%%%%%%%%

% Set up document

\documentclass{beamer}
\usecolortheme{whale}
\setbeamersize{text margin left=5mm,text margin right=5mm}

% Used to create a section slide between section
\AtBeginSection[]{
  \begin{frame}
  \vfill
  \centering
  \begin{beamercolorbox}[sep=8pt,center,shadow=true,rounded=true]{title}
    \usebeamerfont{title}\insertsectionhead\par%
  \end{beamercolorbox}
  \vfill
  \end{frame}
}

% Remove default navigation symbols and add just  page number
\setbeamertemplate{navigation symbols}{} % Clear default navigation
\addtobeamertemplate{navigation symbols}{}{%
    \usebeamerfont{footline}%
    \usebeamercolor[fg]{footline}%
    \hspace{1em}%
    \insertframenumber/\inserttotalframenumber
}


%%%%%%%%%%%%%%%%%%%%%%%%%%%%%%%%%%%%%%%%%%%%%%%%%%%%%%%%%%%%%%%

% Title page

\title{Stroke Audit Machine Learning (SAMueL)}
\subtitle{Learnings from explainable machine learning}


\author{Kerry Pearn\inst{1}, Michael Allen\inst{1,3}, Anna Laws\inst{1}, Richard Everson\inst{3}, Martin James\inst{1,2} }
\institute{\inst{1}University of Exeter Medical School \inst{2}Royal Devon University Healthcare NHS Foundation Trust \inst{3}University of Exeter Institute of Data Science and Artificial Intelligence}

%\institute{Overleaf}
\date{January 2023}


\begin{document}

%\frame{\titlepage}

\begin{frame}
\titlepage


\end{frame}


%%%%%%%%%%%%%%%%%%%%%%%%%%%%%%%%%%%%%%%%%%%%%%%%%%%%%%%%%%%%%%%

\begin{frame}
\frametitle{Machine learning overview}
\begin{center}
\includegraphics[width=0.85\textwidth]{./images/ml_model_high_level}
\end{center}

\small
Machine learning (and nearly all \emph{artificial intelligence}) is based on the simple principle of recognising similarity to what has been seen before.
\vspace{3mm}

We accessed 240,000 emergency stroke admissions in England and Wales over three years. Our machine learning models use XGBoost classification, and are based on all patients who arrive within 4 hours of known stroke onset. 
\end{frame}

%%%%%%%%%%%%%%%%%%%%%%%%%%%%%%%%%%%%%%%%%%%%%%%%%%%%%%%%%%%%%%%

\begin{frame}{Model accuracy, and simplification}

A model with all available 84 features had an ROC AUC of 0.922. A model with 10 features had an ROC AUC of 0.919.

\begin{center}
\includegraphics[width=1.0\textwidth]{./images/01_feature_selection.jpg}
\end{center}

\end{frame}


%%%%%%%%%%%%%%%%%%%%%%%%%%%%%%%%%%%%%%%%%%%%%%%%%%%%%%%%%%%%%%%

\begin{frame}{What do the most thrombolysable patients look like?}

For each hospital we identify the patient with the highest probability of receiving thrombolysis.

\begin{center}
\includegraphics[width=0.6\textwidth]{./images/02a_most_thrombolsyable_violin.jpg}
\end{center}

\end{frame}





%%%%%%%%%%%%%%%%%%%%%%%%%%%%%%%%%%%%%%%%%%%%%%%%%%%%%%%%%%%%%%%

\begin{frame}
\frametitle{Explaining model predictions with SHAP values}

SHAP values show the influence of features (even for \emph{`black box'} models).

\begin{center}
\includegraphics[width=0.90\textwidth]{./images/waterfall.jpg}
\end{center}
\end{frame}


%%%%%%%%%%%%%%%%%%%%%%%%%%%%%%%%%%%%%%%%%%%%%%%%%%%%%%%%%%%%%%%

\begin{frame}
\frametitle{What drives use of thrombolysis across all hospitals?}

\footnotesize{Note: SHAP values here are \emph{log odds}. Each step-change in value of \textpm 1 changes the chances of receiving thrombolysis about 3-fold. (Plots are in order of feature importance.)}

\begin{center}
\includegraphics[width=0.80\textwidth]{./images/03_xgb_10_features_thrombolysis_shap_violin.jpg}
\end{center}
\end{frame}

%%%%%%%%%%%%%%%%%%%%%%%%%%%%%%%%%%%%%%%%%%%%%%%%%%%%%%%%%%%%%%%

\begin{frame}
\frametitle{Hospital SHAP predicts a hospital's predisposition to use thrombolysis}

\begin{center}
\includegraphics[width=1.0\textwidth]{./images/99_twin_correlation_scatter.jpg}
\end{center}
\end{frame}


%%%%%%%%%%%%%%%%%%%%%%%%%%%%%%%%%%%%%%%%%%%%%%%%%%%%%%%%%%%%%%%

\begin{frame}
\frametitle{How general effects may be modified by individual hospitals}

\begin{center}
\includegraphics[width=0.80\textwidth]{./images/12aa_two_way_shap_adjustment.jpg}
\end{center}
\end{frame}

%%%%%%%%%%%%%%%%%%%%%%%%%%%%%%%%%%%%%%%%%%%%%%%%%%%%%%%%%%%%%%%

\begin{frame}
\frametitle{Thrombolysis in subgroups of patients (predicted use in 10k cohort)}

\begin{center}
\includegraphics[width=0.95\textwidth]{./images/15c_modelled_subgroup_violin.jpg}
\end{center}

\scriptsize An \emph{ideal patient} has: Stroke severity NIHSS in range 10-25, Arrival-to-scan time \textless{} 30 minutes, Stroke type = infarction, Precise onset time = True, Prior disability level (mRS) = 0, No use of AF anticoagulants, Onset-to-arrival time \textless{} 90 minutes, Age \textless{}80 years, Onset during sleep = False
\end{frame}

%%%%%%%%%%%%%%%%%%%%%%%%%%%%%%%%%%%%%%%%%%%%%%%%%%%%%%%%%%%%%%%

\begin{frame}
\frametitle{Thrombolysis in subgroups of patients (observed thrombolysis for each team)}

\begin{center}
\includegraphics[width=0.95\textwidth]{./images/15b_actual_subgroup_violin.jpg}
\end{center}


\scriptsize An \emph{ideal patient} has: Stroke severity NIHSS in range 10-25, Arrival-to-scan time \textless{} 30 minutes, Stroke type = infarction, Precise onset time = True, Prior disability level (mRS) = 0, No use of AF anticoagulants, Onset-to-arrival time \textless{} 90 minutes, Age \textless{}80 years, Onset during sleep = False
\end{frame}

%%%%%%%%%%%%%%%%%%%%%%%%%%%%%%%%%%%%%%%%%%%%%%%%%%%%%%%%%%%%%%%

\begin{frame}
\frametitle{Thrombolysis in subgroups of patients (predicted use in 10k cohort)}

    \begin{center}
    \includegraphics[width=1.0\textwidth]{./images/15_10k_subgroup.jpg}
    \end{center}

\footnotesize Note: Subgroups, other than \emph{ideal patients} tend to track each other.  

\end{frame}



\begin{frame}
\frametitle{Thrombolysis in subgroups of patients (observed thrombolysis for each team)}

    \begin{center}
    \includegraphics[width=1.0\textwidth]{./images/15a_actual_subgroup.jpg}
    \end{center}

\footnotesize Note: Subgroups, other than \emph{ideal patients} tend to track each other, but with more noise than predicted thrombolysis use. 

\end{frame}


%%%%%%%%%%%%%%%%%%%%%%%%%%%%%%%%%%%%%%%%%%%%%%%%%%%%%%%%%%%%%%%

\begin{frame}
\frametitle{How would different teams respond to the same patient?}

    \begin{center}
    \includegraphics[width=0.9\textwidth]{./images/shap_waterfall_with_violin.jpg}
    \end{center}

\end{frame}


%%%%%%%%%%%%%%%%%%%%%%%%%%%%%%%%%%%%%%%%%%%%%%%%%%%%%%%%%%%%%%%

\begin{frame}
\frametitle{Summary}
\small

\begin{itemize}
    \item The XGBoost/SHAP model revealed that the odds of receiving thrombolysis:



    \begin{itemize}
        \item Reduced 20 fold over the first 100 minutes of arrival-to-scan time.
        \item Varied 30 fold depending on stroke severity.
        \item Reduced 3 fold with imprecise onset time.
        \item Fell 5 fold with increasing pre-stroke disability
        \item Varied 15 fold between hospitals. 
    \end{itemize}

\item The hospital identification (hospital SHAP value) explained 58\% of the variance in between-hospital thrombolysis use. 

\item Compared with hospitals with higher thrombolysis use, hospitals with lower use were particularly less likely to give thrombolysis to patients with milder strokes, prior disability, or patients with imprecise onset time.
\end{itemize}

\end{frame}

%%%%%%%%%%%%%%%%%%%%%%%%%%%%%%%%%%%%%%%%%%%%%%%%%%%%%%%%%%%%%%%


\end{document}




